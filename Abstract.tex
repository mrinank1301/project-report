\thispagestyle{empty}
\newpage
\cleardoublepage\phantomsection
\addcontentsline{toc}{chapter}{Abstract}\index{Abstract}
\begin{center}
{\Large \bf Abstract}\\
\end{center}
\vspace{10pt}

which has the highest number of smell receptors present in their nasal cavity. The approach also uses the fact that the bloodhound can remember the image odor created in their brain for over 130 miles. In short, whatever the target object being assigned to the bloodhound, intensively searches for it with the help of the odor image created in its brain for that object can remember it for long time period. The approach is implemented using the python platform. The approach is first implemented on the well-known Traveling salesman problem. Traveling salesman problem is one of the standard optimization problems which aim at building the cyclic shortest tour for a salesman visiting each city exactly once. The approach is then tested with the existing results of similar approacheswhich has the highest number of smell receptors present in their nasal cavity. The approach also uses the fact that the bloodhound can remember the image odor created in their brain for over 130 miles. In short, whatever the target object being assigned to the bloodhound, intensively searches for it with the help of the odor image created in its brain for that object can remember it for long time period. The approach is implemented using the python platform. The approach is first implemented on the well-known Traveling salesman problem. Traveling salesman problem is one of the standard optimization problems which aim at building the cyclic shortest tour for a salesman visiting each city exactly once. The approach is then tested with the existing results of similar approacheswhich has the highest number of smell receptors present in their nasal cavity. The approach also uses the fact that the bloodhound can remember the image odor created in their brain for over 130 miles. In short, whatever the target object being assigned to the bloodhound, intensively searches for it with the help of the odor image created in its brain for that object can remember it for long time period. The approach is implemented using the python platform. The approach is first implemented on the well-known Traveling salesman problem. Traveling salesman problem is one of the standard optimization problems which aim at building the cyclic shortest tour for a salesman visiting each city exactly once. The approach is then tested with the existing results of similar approacheswhich has the highest number of smell receptors present in their nasal cavity. The approach also uses the fact that the bloodhound can remember the image odor created in their brain for over 130 miles. In short, whatever the target object being assigned to the bloodhound, intensively searches for it with the help of the odor image created in its brain for that object can remember it for long time period. The approach is implemented using the python platform. The approach is first implemented on the well-known Traveling salesman problem. Traveling salesman problem is one of the standard optimization problems which aim at building the cyclic shortest tour for a salesman visiting each city exactly once. The approach is then tested with the existing results of similar approacheswhich has the highest number of smell receptors present in their nasal cavity. The approach also uses the fact that the bloodhound can remember the image odor created in their brain for over 130 miles. In short, whatever the target object being assigned to the bloodhound, intensively searches for it with the help of the odor image created in its brain for that object can remember it for long time period. The approach is implemented using the python platform. The approach is first implemented on the well-known Traveling salesman problem. Traveling salesman problem is one of the standard optimization problems which aim at building the cyclic shortest tour for a salesman visiting each city exactly once. The approach is then tested with the existing results of similar approaches